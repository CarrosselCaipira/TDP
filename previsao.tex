\section{PREVISÃO DE POSI{\c C}ÃO}

Para possibilitar futuramente a realização de passes e
aumentar a acurácia dos jogadores com relação a bola, está foi
desenvolvido um sistema de previsão de posição da bola e dos demais 
componentes em campo com base no simulador do time Carrossel Caipira.
Como as informações do módulo de previsão serão usadas pelo
módulo da estratégia, a previsão deve ocorrer antes da definição do próximo objetivo. 
Sendo assim, este módulo foi inserido entre o módulo de visão e de estratégia.

Em uma partida existem dois momentos, o de previsão e o de jogo. Enquanto o sistema
está em jogo, todos os módulos trabalham para que cada jogador exerça sua função. Já no
estado de previsão, suas funções ficam restritas a simular o comportamento de todos os
componentes em campo por um determinado espaço de tempo, porém sem que haja o acionamento dos robôs.

Durante a previsão, o módulo de estratégia é executado utilizando as posições 
atuais dos robôs e atualizá-las com base em seus respectivos roteiros. Atualmente é feita uma
previsão de 30 iterações, o que corresponde a 1 segundo, que é o suficiente para melhorar o 
desempenho e continuar tendo um previsão realista.   
