\section{PREVIS{\~A}O DE POSI{\c C}{\~A}O}

Para possibilitar futuramente a realiza{\c c}{\~a}o de passes e
aumentar a acur{\'a}cia dos jogadores com rela{\c c}{\~a}o a bola, est{\'a} foi
desenvolvido um sistema de previs{\~a}o de posi{\c c}{\~a}o da bola e dos demais 
componentes em campo com base no simulador do time Carrossel Caipira.
Como as informa{\c c}{\~o}es do m{\'o}dulo de previs{\~a}o ser{\~a}o usadas pelo
módulo da estratégia, a previs{\~a}o deve ocorrer antes da defini{\c c}{\~a}o do pr{\'o}ximo objetivo. 
Sendo assim, este m{\'o}dulo foi inserido entre o módulo de visão e de estratégia.

Em uma partida existem dois momentos, o de previsão e o de jogo. Enquanto o sistema
está em jogo, todos os m{\'o}dulos trabalham para que cada jogador exer{\c c}a sua fun{\c c}{\~a}o. J{\'a} no
estado de previs{\~a}o, suas fun{\c c}{\~o}es ficam restritas a simular o comportamento de todos os
componentes em campo por um determinado espaço de tempo, porém sem que haja o acionamento dos robôs.

Durante a previs{\~a}o, o m{\'o}dulo de estrat{\'e}gia {\'e} executado utilizando as posi{\c c}{\~o}es 
atuais dos rob{\^o}s e atualiz{\'a}-las com base em seus respectivos roteiros. Atualmente {\'e} feita uma
previs{\~a}o de 30 itera{\c c}{\~o}es, o que corresponde a 1 segundo, que é o suficiente para melhorar o 
desempenho e continuar tendo um previsão realista.   
