%%%%%%%%%%%%%%%%%%%%%%%%%%%%%%%%%%%%%%%%%%%%%%%%%%%%%%%%%%%%%%%%%%%%%%%%%%%%%%%%
%2345678901234567890123456789012345678901234567890123456789012345678901234567890
%        1         2         3         4         5         6         7         8

%\documentclass[letterpaper, 10 pt, conference]{ieeeconf}  % Comment this line out if you need a4paper

\documentclass[a4paper, 10pt, conference]{ieeeconf}      % Use this line for a4 paper
\usepackage[brazil]{babel} % coisas em portugues.
\usepackage[T1]{fontenc}
\usepackage{graphicx} % insercao de figuras
% estilo das fontes das lagendas das figuras
\usepackage[font=scriptsize, justification=centering]{caption}
% funcoes matematicas
\usepackage{amsmath} % assumes amsmath package installed
% Define o caminho das figuras
\graphicspath{{imagens/}}

\IEEEoverridecommandlockouts                              % This command is only needed if 
                                                          % you want to use the \thanks command

\overrideIEEEmargins                                      % Needed to meet printer requirements.

% See the \addtolength command later in the file to balance the column lengths
% on the last page of the document

% The following packages can be found on http:\\www.ctan.org
%\usepackage{graphics} % for pdf, bitmapped graphics files
%\usepackage{epsfig} % for postscript graphics files
%\usepackage{mathptmx} % assumes new font selection scheme installed
%\usepackage{times} % assumes new font selection scheme installed
%\usepackage{amssymb}  % assumes amsmath package installed

\title{\LARGE \bf
O Time Carrossel Caipira de Futebol de Robôs
}

\author{{\centering Danilo Weber Nunes$^{1}$, Rodrigo Bueno dos Reis Rodrigues de Siqueira$^{1}$, Marcelo Nu{\~n}ez$^{1}$, Rafael Tatsuki Takagi$^{1}$,}%
{\authorblockN \centering Mateus B. Santos$^{1}$, Everton Kelvin Francisco$^{2}$, Thiago Menão Mochetti$^{3}$, André Neves Costa da Silva$^{3}$,}%
{\authorblockN \centering Matheus Alves de Sousa Viana$^{3}$}%
%
\thanks{$^{1}$Rodrigo Bueno dos Reis Rodrigues de Siqueira (email: rodrigo.buenorrs@gmail.com),
Marcelo Nuñez (email: marcelo.nunez@hotmail.com), Danilo Weber Nunes (email: danilownunes@gmail.com), Rafael Tatsuki Takagi (email: rafaelttakagi@gmail.com) e Mateus B. Santos (email: mateusbatistasantos@gmail.com) são alunos de Bacharelado em Ciência da Computação.}
\thanks{$^{2}$Everton Kevin Francisco (email: everton\_kelvin@hotmail.com) é aluno de Bacharelado em Sistemas de Informação.}
\thanks{$^{3}$Thiago Menão Mochetti (email: thiagomochetti@gmail.com), André Neves Costa da Silva (email: neves.andre27@gmail.com) e Matheus Alves de Sousa Viana (email: mathvna@gmail.com) são alunos de Engenharia Elétrica da Unesp, campus de Bauru.}
}



%%%%%%%%%%%%%%%%%%%%%%%%%%%%%%%%%%%%%%%%%%%%%%%%%%%%%%%%%%%%%%%%%%%%%%%%%%%%%%%%

\begin{document}
\selectlanguage{brazil}


\maketitle
\thispagestyle{empty}
\pagestyle{empty}

% redefinindo abstract como resumo
\renewenvironment{abstract}{\small \bf {\textit{Resumo}---}}

%%%%%%%%%%%%%%%%%%%%%%%%%%%%%%%%%%%%%%%%%%%%%%%%%%%%%%%%%%%%%%%%%%%%%%%%%%%%%%%%

\begin{abstract}
Este artigo apresenta aspectos de hardware e software do time Carrossel Caipira, que representa o
Departamento de Computa{\c c}{\~a}o da UNESP, campus de Bauru, na modalidade IEEE Very Small Size de futebol de rob{\^o}s. O
Hardware {\'e} composto por tr{\^e}s rob{\^o}s que utilizam Arduino para controle dos motores e recebimento de sinais de r{\'a}dio, que s{\~a}o
enviados por um computador pessoal. O Software, executado neste computador, {\'e} composto por um conjunto de m{\'o}dulos que
inclui: Vis{\~a}o, Estrat{\'e}gia e Controle.
\end{abstract}
 

 \section{INTRODU{\c C}{\~A}O}

O Departamento de Computa{\c c}{\~a}o da Faculdade de Ci{\^e}ncias
da UNESP, campus de Bauru, participa de competi{\c c}ões de
futebol de rob{\^o}s, na modalidade Very Small Size (atualmente
IEEE Very Small), desde 1998, com a realiza{\c c}{\~a}o do 1\textordmasculine
Campeonato Brasileiro de Futebol de Rob{\^o}s -- CBFR 98. A
pesquisa e o desenvolvimento em futebol de rob{\^o}s, atualmente
pelo Grupo de Integra{\c c}{\~a}o de Sistemas e Dispositivos
Inteligentes (GISDI), mant{\'e}m o objetivo de incentivar o uso de
inova{\c c}ões tecnol{\'o}gicas, no campo da rob{\'o}tica, de baixo custo e
com componentes encontrados no mercado nacional. Apesar
de diversos per{\'i}odos de aus{\^e}ncia, o time de futebol da UNESP
de Bauru {\'e} conhecido, desde a primeira edi{\c c}{\~a}o desta
competi{\c c}{\~a}o no Brasil, como Carrossel Caipira devido sua
estrat{\'e}gia de jogo. O projeto atual {\'e} a quinta vers{\~a}o, de rob{\^o}s
desenvolvidos para futebol de rob{\^o}s, com aprimoramentos em
rela{\c c}{\~a}o ao time de 2015.

No ambiente do futebol de rob{\^o}s, nesta categoria, os rob{\^o}s
e a bola s{\~a}o identificados atrav{\'e}s de uma c{\^a}mera utilizada
como vis{\~a}o global, posicionada a 2m sobre o campo e alinhada
ao seu centro, que captura imagens da arena. Estas imagens s{\~a}o
processadas digitalmente obtendo as coordenadas dos rob{\^o}s da bola.
A partir dessas coordenadas, uma estrat{\'e}giaescolhida e transformada em comandos que s{\~a}o enviados aos
rob{\^o}s por r{\'a}dio. Os rob{\^o}s recebem estes comandos e realizam
as a{\c c}ões correspondentes, modificando a posi{\c c}{\~a}o dos
elementos presentes no ambiente real, que ser{\'a} capturado
novamente pela c{\^a}mera. A Fig. 1 apresenta uma ilustra{\c c}{\~a}o do
ambiente do futebol de rob{\^o}s.

% FIGURA 1
\begin{figure}[!htb]
\centering
\includegraphics[scale=0.5]{fut_robos.png}
\caption{Ambiente para futebol de rob{\^o}s. Fonte:www.mecatronicaatual.com.br/secoes/leitura/950 (2013)}
\label{Rotulo}
\end{figure}
%%%

O futebol rob{\'o}tico abrange diversas {\'a}reas do conhecimento.
Na constru{\c c}{\~a}o do rob{\^o} s{\~a}o aplicados conceitos de mec{\^a}nica,
eletr{\^o}nica e sistemas embarcados. Do ponto de vista do
software, executado no computador pessoal, est{\~a}o envolvidos
elementos de processamento de imagens, intelig{\^e}ncia artificial
e teoria de controle. Essa abrang{\^e}ncia faz desta modalidade de
futebol uma ferramenta pedag{\'o}gica com poss{\'i}veis aplica{\c c}ões
no ensino.
Esse projeto busca incentivar e facilitar o desenvolvimento
da rob{\'o}tica, para isso o artigo faz uma apresenta{\c c}{\~a}o das tarefas
realizadas, enfatizando a melhoria aplicada recentemente no
simulador do ambiente para divulgar a efici{\^e}ncia obtida ap{\'o}s
essa modifica{\c c}{\~a}o.
\section{SISTEMA DE SOFTWARE DO FUTEBOL DE ROB{\^O}S}

O sistema deste time pode ser representado
simplificadamente atrav{\'e}s do diagrama apresentado na Fig. 2
que indica as partes principais do processamento. Estas partes
s{\~a}o descritas, juntamente com as intera{\c c}ões entre elas, na
sequ{\^e}ncia. A C{\^A}MERA captura uma imagem do campo que {\'e}
digitalizada por uma placa espec{\'i}fica. Esta imagem {\'e} processada pelo m{\'o}dulo de 
VIS{\~A}O que determinar{\'a} as posi{\c c}{\~o}es atuais dos rob{\^o}s a partir de suas etiquetas coloridas 
e a da bola que possui cor alaranjada. A PREVIS{\~A}O, com base das informações recebidas do m{\'o}dulo da
VIS{\~A}O, define as posi{\c c}{\~o}es mais prov{\'a}veis que os objetos em campo ir{\~a}o 
assumir alguns instantes a frente. Com estas posi{\c c}{\~o}es (presente e futura). O m{\'o}dulo de
ESTRAT{\'E}GIA calcula os locais do campo que os rob{\^o}s do time controlado dever{\~a}o se posicionar. 
Para realizar estes c{\'a}lculos, este m{\'o}dulo faz uso de roteiros, que são específicos para cada rob{\^o}.
Basicamente, um roteiro {\'e} um conjunto de comportamentos espec{\'i}ficos para cara rob{\^o} que faz com que
este assuma uma postura defensiva ou ofenciva durante uma partida. O m{\'o}dulo de CONTROLE, fazendo uso das 
posi{\c c}{\~a}o atuais de cada rob{\^o} (VIS{\~A}O) e de seus objetivos (ESTRAT{\'E}GIA), determina maior
velocidade poss{\'i}vel que um rob{\^o} pode assumir para que este consiga chegar ao seu objetivo e parar,
definindo os valores de cada roda a serem enviados aos rob{\^o}s via r{\'a}dio, fazendo os rob{\^o}s se moverem 
para concluir a estrat{\'e}gia. Todos estes m{\'o}dulos s{\~a}o executados no computador pessoal.

% FIGURA 2
\begin{figure}[!htb]
\centering
\includegraphics[width=240pt, height=160pt]{novo_carrosselTDP.png}
\caption{Diagrama simplificado dos m{\'o}dulos do time Carrossel Caipira.}
\label{Rotulo}
\end{figure}
%%%

\subsection{M{\'o}dulo da Vis{\~a}o}

No ambiente de futebol de rob{\^o}s toda a estrat{\'e}gia e o
controle, tanto de baixo n{\'i}vel quanto de alto n{\'i}vel, s{\~a}o
baseados na interpreta{\c c}{\~a}o das imagens captadas pela c{\^a}mera.
Para que isso seja realizado, etiquetas de cores em destaque
localizadas no topo dos rob{\^o}s identificam cada um deles, em
rela{\c c}{\~a}o a seu time e possivelmente sua fun{\c c}{\~a}o, conforme
demonstra a Fig. 3.

O tempo de execu{\c c}{\~a}o do ciclo de controle do sistema foi
definido pela taxa de aquisi{\c c}{\~a}o de imagens. Como o time
Carrossel Caipira usa c{\^a}meras de v{\'i}deo convencionais, a taxa {\'e}
limitada a 30 quadros por segundo. Portanto, a cada per{\'i}odo de
33 ms, uma nova imagem refletindo o estado atual do campo
torna-se dispon{\'i}vel ao computador para processamento. Cada
uma dessas imagens {\'e} capturada e digitalizada por uma placa
de captura, que disponibiliza, na forma de uma matriz com
dimensões 640x480 pixels com tr{\^e}s canais (componentes Red,
Green, Blue - RGB). Cada um desses pixels deve ser analisado,
quanto a sua cor, para identificar se {\'e} uma cor de import{\^a}ncia
ao sistema, esta t{\'e}cnica {\'e} chamada de segmenta{\c c}{\~a}o de cor.

% FIGURA 3
\begin{figure}[!htb]
\centering
\includegraphics[scale=0.9]{etiqueta.png}
\caption{ Etiqueta de identifica{\c c}{\~a}o do rob{\^o}}
\label{Rotulo}
\end{figure}
%%%

\subsection{M{\'o}dulo de Estrat{\'e}gia}

A estrat{\'e}gia {\'e} o m{\'o}dulo respons{\'a}vel por definir a meta de
cada rob{\^o}. Utilizando as coordenadas atuais e previstas da bola, dos
rob{\^o}s do time e dos advers{\'a}rios este m{\'o}dulo decide qual {\'e} a posi{\c c}{\~a}o 
mais indicada para cada rob{\^o} do time, de tal forma que corresponda ao comportamento que
os rob{\^o}s devem desempenhar na partida. No caso do time Carrossel Caipira
tem-se tr{\^e}s tipos de comportamentos: goleiro, defensor e
atacante.

{\bf Goleiro}: O rob{\^o} {\'e} posicionado na proje{\c c}{\~a}o da posi{\c c}{\~a}o da
bola com a linha defensiva do gol, se a bola vir diretamente na
dire{\c c}{\~a}o da regi{\~a}o do gol. Caso contr{\'a}rio o rob{\^o} se alinha com a
bola na coordenada y do sistema, dessa forma sempre permanecendo em
frente ao gol. Para permitir que o posicionamento seja r{\'a}pido, o rob{\^o} deve
manter sua orienta{\c c}{\~a}o em 90 ou 270 graus, como {\'e} mostrado
na Fig. 4.

% FIGURA 4
\begin{figure}[!htb]
\centering
\includegraphics[scale=0.5]{pos_goleiro.png}
\caption{Comportamento do goleiro. Fonte: Costa et al. [3].}
\label{Rotulo}
\end{figure}
%%%

{\bf Volante}: O rob{\^o} {\'e} posicionado na dire{\c c}{\~a}o da coordenada y
da bola, bloqueando a bola e impedindo seu avan{\c c}o at{\'e} o gol.
Mas deve permitir a bola passar se ela estiver sendo conduzida
pelo atacante. Quando o volante e a bola est{\~a}o em situa{\c c}{\~a}o
favor{\'a}vel no que diz respeito a um poss{\'i}vel ataque ao gol
advers{\'a}rio, ele troca de comportamento com o atacante e tenta
atacar. A situa{\c c}{\~a}o favor{\'a}vel acontece quando o defensor
estiver pr{\'o}ximo da bola em posi{\c c}{\~a}o de ataque.

{\bf Atacante}: Nesta posi{\c c}{\~a}o existem dois tipos de
comportamento: (i) modo de posicionamento: quando o
objetivo {\'e} posicionar o rob{\^o} atr{\'a}s da bola para bloquear algum
advers{\'a}rio ou deixar o rob{\^o} apto para trocar seu modo; (ii)
modo de condu{\c c}{\~a}o, quando o rob{\^o} empurra a bola at{\'e} o gol.

Atrav{\'e}s destes comportamentos {\'e} calculada a posi{\c c}{\~a}o que o
rob{\^o} deve atingir, ou seja, as coordenadas x, y e o {\^a}ngulo que o
rob{\^o} deve alcan{\c c}ar para realizar a estrat{\'e}gia espec{\'i}ficada por seu roteiro.
Esta posi{\c c}{\~a}o {\'e} importante para o m{\'o}dulo de Controle, no que se
refere ao planejamento do caminho {\'o}timo, pois define o fim do
trajeto (o objetivo) escolhido por ele, bem como o {\^a}ngulo desejado de chegada, permitindo
calcular a forma como o rob{\^o} deve chegar l{\'a}.

\subsection{M{\'o}dulo de Controle}

A partir das coordenadas detectadas pelo m{\'o}dulo de Vis{\~a}o
e das coordenadas atribu{\'i}das pelos resultados do m{\'o}dulo de
Estrat{\'e}gia, o m{\'o}dulo de Controle deve determinar as melhores
trajet{\'o}rias e comandos a ser enviados aos rob{\^o}s.

O c{\'a}lculo de uma trajet{\'o}ria deve levar em conta o desvio de
obst{\'a}culos, evitando os outros rob{\^o}s na arena. Este c{\'a}lculo {\'e}
necess{\'a}rio para levar o rob{\^o} da sua posi{\c c}{\~a}o atual at{\'e} a
determinada pela estrat{\'e}gia. No caso do time Carrossel Caipira,
{\'e} empregado um m{\'e}todo chamado de campo potencial. Os
campos potenciais partem da ideia de for{\c c}as imagin{\'a}rias
atuando sobre o rob{\^o}, ideia proposta por Khatib [4], na qual a
''for{\c c}a causada'' pelos obst{\'a}culos {\'e} de car{\'a}ter repulsivo e pela
meta, de car{\'a}ter atrativo. Por{\'e}m a interfer{\^e}ncia das ''for{\c c}as''?
geradas a partir de v{\'a}rios obst{\'a}culos podem produzir locais
{\'o}timos que atrapalham o desempenho do sistema para
encontrar um caminho at{\'e} a meta para o rob{\^o}.

Para evitar esta situa{\c c}{\~a}o, Connoly et al. [5] solucionaram
o problema utilizando fun{\c c}ões harm{\^o}nicas para o c{\'a}lculo do
campo potencial de ambientes nos quais as posi{\c c}ões das
paredes, objetos e metas sejam conhecidas, que {\'e} o caso do
ambiente de futebol de rob{\^o}s. As fun{\c c}ões harm{\^o}nicas
utilizadas s{\~a}o solu{\c c}ões para a equa{\c c}{\~a}o de Laplace (1).
\begin{equation}
\nabla = 0 \ para \ P : R \rightarrow R .
\end{equation}

Assim {\'e} definido um Problema de Valor de Contorno na
regi{\~a}o de atua{\c c}{\~a}o do rob{\^o} utilizando a condi{\c c}{\~a}o de Dirichlet%
\footnote{A condição de contorno de Dirichlet (ou de primeiro tipo) é um tipo
de condição de contorno, nomeada em homenagem a Johann Peter
Gustav Lejeune Dirichlet (1805-1859). Quando aplicada sobre uma
equação diferencial ordinária ou parcial, especifica os valores que uma
solução necessita para tomar-se sobre o contorno do domínio
}
com potencial alto para obst{\'a}culos e potencial baixo para a
meta. Ent{\~a}o, s{\~a}o extra{\'i}das as linhas de for{\c c}a, com base no
gradiente descendente [5][6] do potencial, que direcionam o
rob{\^o} para sua meta, desviando-o de obst{\'a}culos.

Uma vez obtido o campo potencial, tem-se o {\^a}ngulo ideal
que o rob{\^o} deve atingir para se deslocar at{\'e} a meta, que
chamamos de {\^a}ngulo objetivo, basta agora calcular a dire{\c c}{\~a}o e
velocidade de cada motor. 


\section{SIMULADOR E TESTES}

O desenvolvimento e os testes utilizaram o simulador no
time da UNESP-Bauru previamente existente. O simulador
executa o m{\'o}dulo de estrat{\'e}gia e o de controle, oriundos do
software executado para o ambiente real, sem altera{\c c}ões nos
c{\'o}digos. Apenas os arquivos de c{\'o}digos dos dois m{\'o}dulos
(estrat{\'e}gia e controle) da pasta de fontes destinado ao ambiente
real precisam ser transportados para a pasta do simulador,
nenhuma outra altera{\c c}{\~a}o precisa ser realizada. Apesar da
din{\^a}mica ser pouco considerada neste simulador, ele simplifica
a realiza{\c c}{\~a}o de testes dos algoritmos em desenvolvimento, sem
a necessidade da montagem do ambiente real. Uma imagem do
simulador em uma situa{\c c}{\~a}o de jogo pode ser na vista na Fig. 5.
O simulador pode tamb{\'e}m apresentar o campo potencial
gerado pelo m{\'o}dulo de controle, Fig. 6.

% FIGURA 5
\begin{figure}[!htb]
\centering
\includegraphics[scale=0.4]{simulador.png}
\caption{Situa{\c c}{\~a}o representada pelo simulador.}
\label{Rotulo}
\end{figure}
%%%

% FIGURA 6
\begin{figure}[!htb]
\centering
\includegraphics[scale=0.3]{simulador_CP.png}
\caption{Representa{\c c}{\~a}o do campo potencial para a situa{\c c}{\~a}o apresentada na
Fig. 5 considerando o rob{\^o} 1V (volante). Nos quais os quadrados preto, verde
e azul representam respectivamente meta, rob{\^o} e obst{\'a}culo.}
\label{Rotulo}
\end{figure}
%%%

Para que os teste pudessem gerar an{\'a}lises convincentes em
rela{\c c}{\~a}o ao desenvolvimento da estrat{\'e}gia e do controle, foi
necess{\'a}rio acrescentar uma nova funcionalidade que
permitisse dois times jogarem entre si. Para isso foi
desenvolvido no c{\'o}digo do simulador a possibilidade de se
comunicar em rede, de tal forma que houvessem dois times
clientes, cada um com sua estrat{\'e}gia e controle, se
comunicando com o servidor do simulador, que efetiva os
comandos enviados por cada cliente, realizando a
movimenta{\c c}{\~a}o dos rob{\^o}s e da bola virtualmente, e fazendo o
papel da vis{\~a}o ao fornecer o estado de cada rob{\^o} para os times
clientes. Para a comunica{\c c}{\~a}o entre os clientes e o servidor
optou-se por um protocolo de comunica{\c c}{\~a}o simples, o User
Datagram Protocol (UDP).

Desta forma mudou-se a forma como o simulador funciona.
Na Fig. 7 {\'e} apresentado o esquema, de forma simplificada, do
simulador antigo, em que a estrat{\'e}gia recebe o estado do rob{\^o},
calcula o objetivo e envia para o controle. O controle vai
calcular a trajet{\'o}ria e o comando que ser{\'a} enviado para cada
roda, que ser{\'a} enviado para o simulador que efetua a
movimenta{\c c}{\~a}o.

% FIGURA 7
\begin{figure}[!htb]
\centering
\includegraphics[scale=0.5]{esquema_simulador.png}
\caption{Simulador antes das altera{\c c}{\~o}es.}
\label{Rotulo}
\end{figure}
%%%

Na Fig. 8 {\'e} apresentado o esquema do simulador atual, no
qual o simulador torna-se um servidor, que recebe os
comandos de cada roda, efetua-os e em seguida envia o estado
do rob{\^o} para os clientes. Os clientes recebem o estado,
calculam o objetivo atrav{\'e}s da estrat{\'e}gia e os comandos a ser
enviados para o simulador atrav{\'e}s do controle, na sequ{\^e}ncia
esse comando {\'e} enviado para o servidor que efetuar{\'a} a
movimenta{\c c}{\~a}o dos rob{\^o}s. Al{\'e}m disso, tem-se a representa{\c c}{\~a}o
para a conex{\~a}o de um cliente, mas a representa{\c c}{\~a}o {\'e} a mesma
para dois clientes, que {\'e} o caso do futebol de rob{\^o}s.

% FIGURA 8
\begin{figure}[!htb]
\centering
\includegraphics[scale=0.5]{esqumatica_simulador.png}
\caption{ Simulador ap{\'o}s altera{\c c}{\~o}es, representando uma conex{\~a}o.}
\label{Rotulo}
\end{figure}
%%%

Assim, o novo simulador permite intera{\c c}{\~a}o entre dois times
virtuais, levando em considera{\c c}{\~a}o a din{\^a}mica, cinem{\'a}tica e
demais constantes f{\'i}sicas referentes ao rob{\^o} do time Carrossel
Caipira, possibilitando a compara{\c c}{\~a}o atrav{\'e}s de estat{\'i}sticas
que podem ser coletadas durante a execu{\c c}{\~a}o do programa, na
figura 9 temos a imagem do novo simulador.

% FIGURA 9
\begin{figure}[!htb]
\centering
\includegraphics[scale=0.48]{simulador2.png}
\caption{ Simulador ap{\'o}s altera{\c c}{\~o}es, representando uma conex{\~a}o.}
\label{Rotulo}
\end{figure}
%%%

Nesta vers{\~a}o, o simulador procura entender o
comportamento da estrat{\'e}gia e controle estudados, atrav{\'e}s de
situa{\c c}ões de jogo como: placar total, gols contra, p{\^e}naltis
usufru{\'i}do pelo time Carrossel Caipira para o desenvolvimento
das pesquisas envolvendo futebol de cometidos, p{\^e}naltis
convertidos, gols de cada rob{\^o}, n{\'u}mero de free balls e tempo de
jogo. Al{\'e}m disto, {\'e} poss{\'i}vel verificar a evolu{\c c}{\~a}o do placar no
decorrer da simula{\c c}{\~a}o e avaliar o intervalo de confian{\c c}a para
estimar a margem de erro dos dados levantados. A an{\'a}lise
quantitativa dos dados estimula a an{\'a}lise com dados reais feita
no ambiente controlado do laborat{\'o}rio que complementa o
ambiente instrumental rob{\^o}s. 

\section{PREVISÃO DE POSI{\c C}ÃO}

Para possibilitar futuramente a realização de passes e
aumentar a acurácia dos jogadores com relação a bola, está foi
desenvolvido um sistema de previsão de posição da bola e dos demais 
componentes em campo com base no simulador do time Carrossel Caipira.
Como as informações do módulo de previsão serão usadas pelo
módulo da estratégia, a previsão deve ocorrer antes da definição do próximo objetivo. 
Sendo assim, este módulo foi inserido entre o módulo de visão e de estratégia.

Em uma partida existem dois momentos, o de previsão e o de jogo. Enquanto o sistema
está em jogo, todos os módulos trabalham para que cada jogador exerça sua função. Já no
estado de previsão, suas funções ficam restritas a simular o comportamento de todos os
componentes em campo por um determinado espaço de tempo, porém sem que haja o acionamento dos robôs.

Durante a previsão, o módulo de estratégia é executado utilizando as posições 
atuais dos robôs e atualizá-las com base em seus respectivos roteiros. Atualmente é feita uma
previsão de 30 iterações, o que corresponde a 1 segundo, que é o suficiente para melhorar o 
desempenho e continuar tendo um previsão realista.   


\section{O HARDWARE DO ROBÔ}

Após o contato com outras equipes e pesquisa em relação aos materiais, decidiu-se por reformular a estrutura dos robôs. Visando a simplificação das montagens, bem como agilidade na manutenção, os robôs adquiriram um escopo modular. Dessa forma, cada conjunto de componentes responsável por um setor do robô é acoplado a um dos módulos separadamente. Assim, caso algum dos módulos apresente problemas de execução, ele pode ser trocado por outro igual reserva, sem que haja excessiva perda de tempo. Inicialmente, cogitou-se a existência de três módulos: os motores junto das rodas; a bateria; e a placa controladora com os componentes discretos. O módulo da placa controladora, que contém, entre outros componentes, o Arduino Nano[6], é agora livre de fios e componentes soldados diretamente ao fenolite. Novamente em busca de agilidade na manutenção, somente soquetes e caixas de pinos foram soldadas diretamente na placa, de modo que os componentes são removíveis e de fácil substituição. A bateria, que é empregada agora no lugar das pilhas AA de versões anteriores, ocupa sozinha, junto de um pequeno circuito de verificação de carga, o segundo módulo. Isso ocorre devido ao trabalho necessário para carregá-la. Diferentemente das pilhas, onde bastava uma fonte de alimentação para recarregar o robô, a bateria de lítio exige um circuito próprio externo ao robô para carregar adequadamente. O circuito verificador de carga é responsável por impedir que a bateria chegue a níveis muito baixos, o que prejudicaria seu desempenho. O último dos módulos é composto pelos motores e rodas. A única diferença desse sistema para o anterior é o diâmetro das rodas, que cresceu 2mm, e seu acabamento.

% FIGURA
\begin{figure}[!htb]
\centering
\includegraphics[scale=0.7]{hardware_robo.png}
\caption{Esquema dos componentes eletrônicos usados em cada robô.}
\label{fig:hardware_robo}
\end{figure}

O robô foi construído tem um Arduino Nano[6] como módulo de controle. Este módulo recebe, via um transceptor de rádio NRF24L01, os comandos de velocidade desejados para cada uma das rodas. O Arduino aciona os motores utilizando modulação por largura de pulso PWM (Pulse-Width Modulation) através de uma placa equipada com o componente por um circuito impresso L293.

A rotação das rodas, e consequentemente o deslocamento do robô, são determinados por sensores TCRT5000, que codificam o movimento de dentes reflexivos fixados às rodas em pulsos enviados ao Arduino. Esses pulsos são contados e servem de realimentação ao acionamento direcionado a cada motor. Um esquema representativo da eletrônica do robô pode ser apreciado na Fig. \ref{fig:hardware_robo} .

Do lado do computador pessoal, o envio dos comandos também ocorre através de um circuito controlado por um Arduino Nano[6] conectado a uma interface USB. Este Arduino recebe os comandos a ser enviados aos robôs e os transmite por meio de um transceptor de rádio NRF24L01 idêntico ao utilizados nos robôs.



%%%%%%%%%%%%%%%%%%%%%%%%%%%%%%%%%%%%%%%%%%%%%%%%%%%%%%%%%%%%%%%%%%%%%%%%%%%%%%%%

\begin{thebibliography}{99}

\bibitem{c1} UMBAUGH, S. E.; Digital Image Processing and Analysis: Human and Computer Vision Applications with CVIPtools, Second Edition, CRC Press,November 19, 2010
\bibitem{c2} ALVES, S. F. R.; FERASOLI FILHO, H.; PEGORARO, R.; CALDEIRA, M. A. C.; ROSARIO, J. M.; YONEZAWA, W. M.; Proposal of educational environments with mobile robots; Conference on Robotics, Automation and Mechatronics (RAM), 2011.
\bibitem{c3} COSTA, A. H. R. ; PEGORARO, R. . Construindo rob{\^o}s aut{\^o}nomos para partidas de futebol: o time GUARAN{\'a}. Controle \& Automa{\c c}{\~a}o, Campinas, SP, v. 11, n. 2, p. 141-149, 2000.
\bibitem{c4} KHATIB, O.; Command Dynamic dans l'Espace Op{\'e}rationnel des Robots Manipulateurs en Pr{\'e}sence d'Osbtacles. PhD thesis, Ecole Nationale Sup{\'e}rieure de l'A{\'e}ronautique et de l'Espace, Toulouse. em Franc{\^e}s. 1980.
\bibitem{c5} CONNOLLY, C. I., BURNS, J. B.; WEISS, R. Path planning using laplaces equation. In IEEE International Conference on Robotics and Automation, 1990.
\bibitem{c6} CONNOLLY, C. I.; GRUPEN, R. A. On the application of harmonic functions to robotics. Journal of Robotic Systems, 10:931-946, 1993. Placa Arduino Nano 3.0. Dispom{\'i}vel em: <http://arduino.cc/en/Main/ArduinoBoardNano> Acesso em 3 de agosto de 2014.
\bibitem{c7} SANTOS, Rafael, Mecanismo de Previs{\~a}o de Posi{\c c}{\~a}o para o Time de Futebol de Rob{\^o}s da UNESP, Trabalho de Conclus{\~a}o de Curso do Curso, UNESP - Universidade Estadual Paulista ''J{\'u}lio de Mesquita Filho'', Bauru, Brasil, Fev. 2017.

\end{thebibliography}



\end{document}
